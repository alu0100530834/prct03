\documentclass[a4paper,12pt]{article}
\usepackage[utf8]{inputenc}
\begin{document}
Si simplemente se desea escribir texto normal en Latex,
sin complicadas f\'ormulas matem\'aticas o efectos especiales
como cambios de fuente, entonces simplemente tiene que escribir
en espa\~nol normalmente.
$\\par$
Si desea cambiar de párrafo ha de dejar una linea en blanco o bien
utilizar el comando $\\par$.
$\\par$
No es necesario preocuparse de la sangria de los párrafos:
todos los párrafos se sangrarán automaticamente con la excepción 
del primer parrafo de una sección.

Se ha de distinguir entre la comilla simple 'izquierda'
y la comilla simple 'derecha' cuando se escribe en el ordenador.

En el caso de que se quieran utilizar comillas dobles se han de 
escribir dos caracteres 'comilla simple' seguidos, esto es,
"comillas dobles".

También se ha de tener cuidado con los guiones: Se utiliza un unico
guión para la separación de silabas, mientras que se utilizan 
tres guiones seguidos para producir un guión de los que se usan
como signo de puntuación --- como en esta oración.
\begin{thebibliography}{00}
 \bibitem{Lam:86}
   Lamport,Leslie.
   TLA in pictures.
   \emph{IEEE Transactions on Software Engineering},
   21(9), 768-775
   (1995)
\end{thebibliography}   
\end{document}


